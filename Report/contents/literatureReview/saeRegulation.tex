\section{The \textit{SAE J2522} regulation}

		\textit{SAE International} was founded in 1905 and the acronym SAE stands for Society of Automotive Engineers. Nowadays their emphasis is on transports industries, such as automotive, aerospace and commecial vehicles. One of their main activity is providing parameters and regulations of quality and safety standards for the industry. One great example of their operation is the SAE has long provided standards for horsepower rating.
		\par
		More related to this project is the \textit{SAE J2522}, entitled \textit{Dynamometer Global Brake Effectiveness}, at the beggining it already states the it's utility with the following:

		\say{The SAE Brake Dynamometer Test Code Standards Committee considers this standar useful in supporting the technological efforts intended to improve motor vehicle braking systems overall performance and safety}

		\cite{saej2522}. This regulation was developed to be used in conjecture with other test standards in other to address the friction of a certain material to check it's adequacy for a certain application. It is important to state that this paper is based on the \textit{SAE J2522}, it is not a faithful application of the standard though. This paper is more concerned about the settings of the tests mentioned on the regulation rather than the formulas and criteria for a materials engineering analysis.
		\par
		All the tests mentioned on the regulation can generalised on repetitive cycles of accelerating the rotor to a specified speed and appling brake force (may vary along the test) until the rotor reaches a lower limit of speed. On the regulation, sometimes the desacceleration ratio is also defined but not always. Initial temperature is also defined, some tests can only be perfomed if the brake parts are under a certain temperature.

