\subsection{Microcontroller}

A microcontroller is a compact computer on a single integrated circuit chip, in most cases a microcontroller (also refered by the acronym MCU) includes a processor, volatile and non-volatile memory, input/output ports and other peripherals. The great thing about the microcontrollers is their low cost, many small appliances that does not require a powerful hardware are only economically viable because of those devices. The components a microcontroller has may vary, it is a responsability of the project designer to decide the microcontroller that has the best fit (technically and economically) for the project.
	\par
	Microcontrollers differ from microprocessors only in one thing, MCUs can be used standalone while microprocessors need other peripherals to be used. By reducing the size and cost compared to a design that uses a separate microprocessor, memory, and input/output devices, microcontrollers make it economical to digitally control even more devices and processes. \say{A microprocessor can be considered the heart of a computer system, whereas a microcontroller can be considered the heart of an embedded system}\cite{mcuDef}.
	\par
	A great thing about microcontrollers is that they must provide provide real-time response to events, so for instrumentation they are crucial. With them it is possible to acquire signals with good sampling rates without loss of relevant information. It is common in electronic instrumentation to use MCUs to handle the events that have real-time constains and use more sophisticated hardware solutions to manage and process the acquired data later \cite{bartz2004data}.
