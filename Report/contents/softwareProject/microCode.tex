\section{Microcontroller Code} \label{sec:microCode}

	\subsection{Microcontroller programming basics}
		Microcontrollers were traditionally programmed using Assembly language, nowadays we have a different scenario though. With the advancement of compilers, today it is most common to write the code for a microcontroller in C language and let the compiler translate it to a binary assembly format \cite{Mazidi:2010:AME:1824214}. A convinient reason to use C is that it is one of the most efficient programming languages in terms of execution speed, this happens because it was designed to efficiently map typical machine instructions \cite{kernighan2006c}, so considering real-time constrains and other execution constrains in microcontrollers it is a excelent fit.
		
		\par
		A microcontroller code is composed basically of two parts:
		
		\begin{itemize}
			\item \textit{Setup: } This part of the code is only executed once, as the name may indicate, it is used to set properties, configure timers, inputs and other features of the hardware.
			\item \textit{Loop: } This part of the code is executed continuously or until some condition is reached. 
		\end{itemize}
		
		\par
		
		Other important component of a microcontroller code are interruptions. It is possible to interrupt the standard execution of a program when a event happens, or as it is more common to say, when a event triggers a interruption. This event may be a timer overflow, a event triggered by an input change among other things. Interrupt routines are really useful when working with intrumentation and timers, because using interruptions it is feasible to meet real-time requirements in a project \cite{mukaro1999microcontroller}.
		
	\subsection{Microcontroller Code Map}
	
	The Figure \ref{fig-microCodeMap} show a functional map of the microcontroller code that can be found entirely in the annex of this paper in Section \ref{sec:microCode}.
	
	\begin{figure}[htbp]
		\centering
		\includegraphics[scale=0.8]{figuras/fig-microCode}
		\caption{Microcontroller Code Map \cite{microCodeMap}}
		\label{fig-microCode}
	\end{figure}
	
	
	As soon as the microcontroller is turned on it enters in the \textit{Setup}, on this part the following things are setted:
	\begin{itemize}
		\item \textit{Timer_1: } Timer 1 max count value is setted and a interruption service routine is appointed.
		\item \textit{Timer_2: } Same thing done for Timer_1 is done for Timer_2.
		\item \textit{Serial Port: } The serial port baud rate is defined and the serial port is opened.
		\item \textit{Port definitions: } The I/O ports are defined as inputs (high impedance) or as outputs (low impedance).
	\end{itemize}
	
	After setting up the code enters in a state in which it waits for a \textit{Start Command} from the computer. This command starts Timer_1 and makes the code enter the \textit{Acquisition State}. Everytime Timer_1 finishes its count a ISR will be triggered, all the analog inputs will be read (sampled) and stored followed by the timer restarting its counting process. The \textit{Start Command} also starts Timer_2, this timer operates the same way as Timer_1, the difference is that its ISR will only toggle the acquistion LED state.
	\par
	In the acquistion state, if a \textit{Digital Output Command} is received the code will decrypt this command (this is actually a group of possible commands) and set the digital outputs according to the decrypted message.
	\par
	A received \textit{Speed Command} instruction will make the code behave in a similar way than \textit{Digital Output Command}, code will decript the command (it also is a group of possible commands) and output a corresponding PWM value in the analog output port in order to control the electric motor speed.
	\par
	During all the \textit{Acquisition State}, everytime Timer_1 ISR does a presetted number of reads, the code makes an average of thoose reads and will send them through the serial port to the computer. After that the code will reset the number of read samples and start to count samples again. Another useful feature is that during all the \textit{Acquisition State} the code is also continuosly reading the microcontroller digital inputs, this inputs are used to detect if a sensor is not connected, if any of the sensors is considered to be disconnected instead of sending the analog reading from the respective sensor input, the code will send a "-1" to the computer. This way the computer software that the one responsible for treating this event and take the appropriate action.
	\par
	The code will exit the \textit{Acquisition State} if the computer sends a \textit{Stop Command}. This will also set all digital outputs to a low logic level and will disable both timers interruption service routines. The code will go back to the \textit{Wait for Command} state and wait for a \textit{Start Command} to restart acquisition. 
	\par
	This code architecture was designed in order to make the microcontroller code as simples as possible in order to make execution faster to avoid jeopardizing the real-time constrain. The same code is used during brake tests, during acquisition and any other process. As mentioned before, all the heavy data processing is done by the computer software.