\section{Microcontroller Software}

	\subsection{Microcontroller programming basics}
		Microcontrollers were traditionally programmed using Assembly language, nowadays we have a different scenario though. With the advancement of compilers, today it is most common to write the code for a microcontroller in C language and let the compiler translate it to a binary assembly format \cite{Mazidi:2010:AME:1824214}. A convinient reason to use C is that it is one of the most efficient programming languages in terms of execution speed, this happens because it was designed to efficiently map typical machine instructions \cite{kernighan2006c}, so considering real-time constrains and other execution constrains in microcontrollers it is a excelent fit.
		
		\par
		A microcontroller code is composed basically of two parts:
		
		\begin{itemize}
			\item \textit{Setup: } This part of the code is only executed once, as the name may indicate, it is used to set properties, configure timers, inputs and other features of the hardware.
			\item \textit{Loop: } This part of the code is executed continuously or until some condition is reached. 
		\end{itemize}
		
		\par
		
		Other important component of a microcontroller code are interruptions. It is possible to interrupt the standard execution of a program when a event happens, or as it is more common to say, when a event triggers a interruption. This event may be a timer overflow, a event triggered by an input change among other things. Interrupt routines are really useful when working with intrumentation and timers, because using interruptions it is feasible to meet real-time requirements in a project \cite{mukaro1999microcontroller}.
		
	\subsection{Microcontroller Code Map}
	
	The Figure \ref{fig-microCodeMap} gives a overral idea of the microcontroller code that can be found in ............
	
	
	As soon as the microcontroller is turned on it enters in the \textit{Setup Function}, on this function the following things are setted:
	\begin{itemize}
		\item \textit{Timer 1: } Timer one is used for controlling acquisition sampling rate, i.e., the interval between each acquisition. Each moment that this timer overflow its count, a interruption is triggered and the analog inputs are read (sensor signals) and stored. In the setup block only the timer clock frequency and the maximum value are defined.
		\item \textit{Timer 2: } This timer is used only to control the blinking frequency of a LED used to indicate if a execution is happening.
		\item \textit{Serial Port: } The serial port baud rate is defined and the serial port is opened.
		\item \textit{Port definitions: } The I/O ports are defined as inputs (high impedance) or as outputs (low impedance).
	\end{itemize}
