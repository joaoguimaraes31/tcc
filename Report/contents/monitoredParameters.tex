\section{Monitored parameters}
		As metioned before this paper will be based in the \textit{SAE J2522} regulations, this regulation says that to evaluate the efficiency of a brake system it is mandatory to monitor temperature on the brake pads, the pressure applied on the disk and the speed of the rotor throughout all the process. Monitoring the vibration is not mandatory but has some advantages.

	\begin{itemize}
		\item\textit{Temperature of brake pads: } During all test it is mandatory to have full knowledge of the temperature of the break pads, firstly because of security reasons (there is upper limit for temperatura in any system) and also because of the wear of parts that is related to temperature.
		\item\textit{Pressure applied on the disks: } Knowing the magnitude of this force means being able to relate the pressure applied and the deceleration, knowing how the pressure applied increases the temperature of the pads and evaluate how this promotes wear of the parts.
		\item\textit{Rotation speed: } Without knowing how the speed of the rotor varies over time it would be impossible to determine the acceleration and deceleration rates among many other issues.
		\item\textit{Vibration: } As mentioned before this is not mandatory but rather interesting, measuring vibration makes it possible to determine how the extensive use can wear out the parts and reduce stiffness among other properties. Also it is natural that the system will vibrate during braking, minimal vibration or too much vibration can indicate a fault that on the future could damage the system.
	\end{itemize}

